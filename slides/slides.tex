\documentclass{beamer}
\usepackage{amsmath}
\usepackage{amssymb}
\usepackage{bm}
\usepackage{hyperref}
\usepackage{csquotes}
\usepackage{graphicx}

\DeclareSymbolFont{matha}{OML}{txmi}{m}{it}% txfonts
\DeclareMathSymbol{\varv}{\mathord}{matha}{118}
\usetheme{metropolis}

\graphicspath{ {pictures/Table_4_1.png}{pictures/Table_4_2.png}{pictures/Figure_3.png} }

\usefonttheme{serif} % default family is serif

\setbeamertemplate{section in toc}[sections numbered]
\setbeamertemplate{subsection in toc}[subsections numbered]

\title{Macroeconomics 3 Presentation}
\subtitle{Article review :}
\author{GUGELMO CAVALHEIRO DIAS Paulo \\ SHARMA Vivan}
\institute{Sciences Po}
\date{\today}

\begin{document}
\begin{frame}
    \titlepage
\end{frame}


\section{Introduction}

\begin{frame}{\secname}
    Study objects of the paper : 
    \begin{itemize}
        \item Contribution of Natural Hazards to Wealth Inequality
        \item Contribution of Insurance Policies to Wealth Inequality
    \end{itemize}
    Focus on the empirical facts : 
    \begin{itemize}
        \item The interaction between natural hazard and pre-existing social disparities
        \item The effect of private and public insurance
    \end{itemize}
\end{frame}

\begin{frame}
    \frametitle{Outline}
    \tableofcontents[hideallsubsections]
\end{frame}

\section{Study Design}
    \begin{frame}{\secname}
        \begin{itemize}
            \item Explaining Wealth with Socioeconomic Variables
            \item Datasets 
        \end{itemize}
    \end{frame}

\subsection{Explaining Wealth with Socioeconomic Variables}
\begin{frame}{\subsecname}
    What is the effect of natural hazard on wealth inequalities ? 
    How does this effect vary in function of socioeconomic variables associated to an individual ? 

    \begin{equation*}
        \widehat{wealth} = \widehat{\alpha} + \widehat{\beta} \cdot \log(\text{natural hazard damages}) + \widehat{\gamma} \cdot X
    \end{equation*}

    With $X$ socioeconomic variables (at the individual, local, and county levels) :
    Race, Foreign Born, Education, Married, Homeownership, Socioeconomic Status of the Neighborhood, etc...
    (discussed below)
\end{frame}

\subsection{Datasets}
\begin{frame}{\subsecname}
\begin{itemize}
    \item \textbf{Panel Study Income Dynamics (PSID)}
    \item \textbf{Spatial Hazard Events and Losses Database (SHELD)}
    \item \textbf{Census} 
    \item \textbf{Federal Emergency Management Agency (FEMA) Projects Summary (2016)}
\end{itemize}
\end{frame}

\begin{frame}{\subsecname}
    \textbf{Panel Study Income Dynamics (PSID)}
    \begin{itemize}
        \item From 1968 to now, with a modification of followed individuals in 1999.
        \item Used sample : from 1999 to 2013, with a two-years interview period. 
        \item Use of the restricted-access, geocoded version of the survey, with information on the neighborhood of the respondents.
    \end{itemize}
\end{frame}

\begin{frame}{\subsecname}
    \textbf{Spatial Hazard Events and Losses Database (SHELD)}
    \begin{itemize}
        \item Maintained by the Hazards and Vulnerability Research Institute (HVRI) from 1960 to now.
        \item They collect information on 18 types of natural hazards and their associated fatalities 
        and property damages. 
    \end{itemize}
\end{frame}

\begin{frame}{\subsecname}
    \textbf{Census}
    \\ Three datasets, from two different census : 
    \begin{itemize}
        \item 2000 U.S Decennial Census Long Form
        \item 2006-2010 and 2011-2015 American Community Survey 5-year Summary Files
    \end{itemize}
    Use : 
    \begin{itemize}
        \item Socioeconomic status of neighborhoods.
        \item Total population. 
        \item Urban / rural status. 
    \end{itemize}
\end{frame}

\begin{frame}{\subsecname}
    \textbf{Federal Emergency Agency (FEMA) Projects Summary (2016)}
    \begin{itemize}
        \item Immediate Needs Funding (INF)
        \item Individual and Household Program (IHP)
        \item local husing vouchers, temporary units, financial grants, replaceent of property, etc...
    \end{itemize}
\end{frame}

\section{Wealth Inequality : the general effects}

    \begin{frame}{\secname}
        \includegraphics[totalheight=7cm,width=1\textwidth]{pictures/Table_4_1.png}
    \end{frame}

    \begin{frame}{\secname}
        \includegraphics[width=1\textwidth]{pictures/Table_4_2.png}
    \end{frame}

    \begin{frame}{\secname}
        Wealth : Definition of PSID, sum of saving accounts, checking accounts, real estate holdings, equity, vehicles,
        farms, businesses, stocks, annuities / IRAs, minus all reported debts.
        \\ Dollars : logarithm, standardizing all values to 2012 dollars. 
    \end{frame}


\section{Wealth Inequality : a concrete example}
    \begin{frame}{\secname}
        \includegraphics[totalheight=7cm,width=1\textwidth]{pictures/Figure_3.png}
    \end{frame}

\section{Conclusion}
    \begin{frame}{\secname}
        In a broader scale, natural hazards shocks do provoke an increase in wealth disparities. 
        This paper points out \underline{the role of}
        \underline{natural hazards and insurance in the increase of the wealth gap}.

        In front of environmental risks, agents are differently exposed not only due to budget constraints,
        but also due to \underline{lacking} \underline{insurance schemes that would provide a better and equitable}
        \underline{coverage}.
        
        The authors call for further research on the precise mechanisms that make natural hazards affect
        wealth inequalities. Also, they underly the need for 
        \underline{a reform of the current insurance system, that 
        is too much based} \underline{on wealth and not on vulnerability}.
    \end{frame}

\end{document}